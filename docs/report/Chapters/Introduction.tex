% Chapter Template

\chapter{Introduction}\label{Intro}   % Change X to a consecutive number; for referencing this chapter elsewhere, use \ref{Intro}

%----------------------------------------------------------------------------------------
%	SECTION 1
%----------------------------------------------------------------------------------------

\section{Epigenetics}

The genetic material is known to be modified during the life of an organism, possibly causing modifications in gene behavior. Far is known about mutations being a mayor player in genetic variation and the profiling of these variations has proved to be highly useful when studying diseases and evolutionary theory [GWAS]. In eukaryotes, there are in addition mechanisms in which the DNA can be modified without altering the molecular sequence, so called epigenetic mechanisms. As Conrad Waddington, who coined the term ``epigenetics'', defined: ``it is the branch of biology which studies the causal interactions between genes and their products, which bring the phenotype into being'' [Waddington]. Such definition led to categorizing as epigenetics all biological phenomena which correlated the genetic material with the genetic products and were not explained entirely by the classic genetic studies. Further studies have revealed that epigenetic mechanisms can be modulated in response to external stimuli [CITATION], entailing an overlay between DNA  and environment for the cells and organisms. Moreover, the epigenetic mechanisms behavior can vary for different stages of cell development [CITATION], environmental changes [CITATION] or disease [CITATIONS]. In the same way genetic variability can be profiled, it is possible to decipher shared patterns for the epigenetic modifications in different scenarios and types of tissue.

\medskip

Due to the latest growth on research efforts and resources about the topic, we were able to characterized the inheritance of gene expression patterns not explained by the encoded information in the DNA sequence but through epigenetic modifications. High-throughput technologies such as Chromatin Inmunoprecipitation next-generation sequencing (ChIP-seq) or Whole-Genome Bisulfite sequencing (WGBS), allowed us to obtain an incredibly vast amount of information on epigenetic marks throughout the genome. It was then possible to determine the epigenome profile with several chromatin states that enhance or repress the gene activity in a local and cell-specific manner. These ``epigenomic profiling'' helped the understanding of cell differentiation, where even though the vast majority of cells in a multicellular organism share an identical genotype, organismal development generates a diversity of cell types with disparate, yet stable, profiles of gene expression and distinct cellular functions. More specifically, epigenetics may be defined as ``the study of any potentially stable and, ideally, heritable change in gene expression or cellular phenotype that occurs without changes in Watson-Crick base-pairing of DNA'' [CITATION Gol07].


%-----------------------------------
%	SUBSECTION 1
%-----------------------------------
\subsection{Subsection 1}

Nunc posuere quam at lectus tristique eu ultrices augue venenatis. Vestibulum ante ipsum primis in faucibus orci luctus et ultrices posuere cubilia Curae; Aliquam erat volutpat. Vivamus sodales tortor eget quam adipiscing in vulputate ante ullamcorper. Sed eros ante, lacinia et sollicitudin et, aliquam sit amet augue. In hac habitasse platea dictumst.

%-----------------------------------
%	SUBSECTION 2
%-----------------------------------

\subsection{Subsection 2}
Morbi rutrum odio eget arcu adipiscing sodales. Aenean et purus a est pulvinar pellentesque. Cras in elit neque, quis varius elit. Phasellus fringilla, nibh eu tempus venenatis, dolor elit posuere quam, quis adipiscing urna leo nec orci. Sed nec nulla auctor odio aliquet consequat. Ut nec nulla in ante ullamcorper aliquam at sed dolor. Phasellus fermentum magna in augue gravida cursus. Cras sed pretium lorem. Pellentesque eget ornare odio. Proin accumsan, massa viverra cursus pharetra, ipsum nisi lobortis velit, a malesuada dolor lorem eu neque.

%----------------------------------------------------------------------------------------
%	SECTION 2
%----------------------------------------------------------------------------------------

\section{Non-negative Matrix Factorization}

Sed ullamcorper quam eu nisl interdum at interdum enim egestas. Aliquam placerat justo sed lectus lobortis ut porta nisl porttitor. Vestibulum mi dolor, lacinia molestie gravida at, tempus vitae ligula. Donec eget quam sapien, in viverra eros. Donec pellentesque justo a massa fringilla non vestibulum metus vestibulum. Vestibulum in orci quis felis tempor lacinia. Vivamus ornare ultrices facilisis. Ut hendrerit volutpat vulputate. Morbi condimentum venenatis augue, id porta ipsum vulputate in. Curabitur luctus tempus justo. Vestibulum risus lectus, adipiscing nec condimentum quis, condimentum nec nisl. Aliquam dictum sagittis velit sed iaculis. Morbi tristique augue sit amet nulla pulvinar id facilisis ligula mollis. Nam elit libero, tincidunt ut aliquam at, molestie in quam. Aenean rhoncus vehicula hendrerit.
