% Chapter Template

\chapter{Discussion}\label{Discussion} % for referencing this chapter elsewhere, use \ref{Discussion}

The work in this master thesis has been carried out during the lapse of 4 months. Nevertheless, we have been able to find combinatorial patterns between epigenetic modifications, drawn from the raw reads of ChIP-seq experiments. Moreover, we have been able to correlate the different combinatorial patterns generated for three cancer tissues (Hela-S3, HepG2 and K562) and compare them to previous studies on IMR90 cell line. We have also explored the distribution of the signatures along chromosome 4. The findings in this study suggest that it may be possible to generalize the epigenetic signatures to different tissues.

\medskip

If the time was not limiting, it would be interesting to go deeper into the genomic location of the signatures. It would be possible to do an enrichment study using the epigenetic signatures. Therewith, we would be able to relate the epigenetic signatures with specific genes, and not just gene activity. Another interesting path would be to use the epigenetic signatures in a machine learning model, which has been found to be effective in previous studies. From the experience of this master thesis, I feel that it would be interesting to extend on the signature comparison by focusing on the differences between them. In future work, it should be interesting to explore the possibility of relating the signature differences to the differences in the metabolic traits between healthy samples and diseases. More and more studies emphasize the influence of the epigenetic profiles in multiple diseases such as cancer, Alzheimer or schizophrenia.
